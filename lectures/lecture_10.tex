\section{Web Vulnerabilities}
\paragraph{What is cross-site scripting?}
Cross-siste scripting (XSS) allows an attacker to inject arbitrary HTML and JavaScript code. The malicious payload runs in the browser of legitimate users. An import thing to notice is that XSS attacks target application users, not the server itself. XSS attacks are based on the implicit notion of trust that exists between a browser and a server. In general an XSS vulnerability allows the attacker to inject content on a webpage. An XSS attack could disclose cookies that are linked to a session of a user, disclose end user files or redirect the user to another site. It is also possible to change the content of the webpage, or the information on it.

\paragraph{Why does XSS work?}
Clients tend to trust servers, users not generally not paranoid and do not use high levels of protection. Another aspect is that the scripting languages have been deeply embedded into web browsers, and made it extremely powerful. The entire core of the problem is input-sanitation, a user is allowed to pass badly or unsanitized input to a web-application which is then presented to another user.

\paragraph{What are the three types of XSS attacks?}
Reflected: where the malicious string originates from the victim's request. The attacker somehow tricks the user in sending the forged input to the server. A possible scenario is to get a victim clicking on the link containing the XSS attack by sending it to them by email.

Stored: where the malicious string originates from the website's database. This XSS variant is stored on the remote server, for example in a forum thread. Whenever the user retrieves a certain webpage the malicious content is delivered to their browser.

DOM-based: where the vulnerability is in the client-side code rather than the server-side code. The malicious script is embedded in the original code of the page. It gets executed as embedded in the actual legitimate code of the returned page. The legitimate script relies on user input to add HTML content to the page, e.g. additional children, nodes in the DOM, innerHTML is used for this. It ultimately leads to the execution of the added code in the page via innerHTML. The code is executed after the main page is loaded.

\paragraph{What is an SQL injection?}
A web application uses SQL DBs to store information, for example user accounts and credentials. SQL is an interpreted language, it can be used to read, update, add and delete database information. User-supplied information is passed from client (browser) to web-application via e.g. a GET request. The SQL injection is crafting user-supplied input so to execute database queries beneficial to the attacker's project. SQL injections is a very widespread bug, it occurs due to absence of proper input-sanitation. 