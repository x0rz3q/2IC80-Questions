\section{Hacking a Human}
\paragraph{What is social engineering?}
Social engineering is a wide set of attacks that exploit the human nature to usually breach data confidentiality. Social engineering identifies a set of techniques that attack weaknesses in human psychology. The final goal is to persuade a human being in performing actions solicited by the attacker.

\paragraph{Explain the likelihood model (ELM)}
ELM describes the ways humans change their attitudes or decide to perform actions they would not perform without external stimuli. There are two router to persuasion. Namely, the central route and the peripheral route.

The central route is that stimuli are weighted by the subject and the final decision is carefully elaborated. There are high amount of cognitive effort, associated with "rational perfectly informed decisions" in economics. Persuasion happens through careful elaboration of information.

The peripheral route does communication that typically does not result in careful cognitive effort in understanding the message. The subject is convinced by under-analysing apparently relevant "cues" that are in reality unrelated to the subject matter. Persuasion happens through "adjunct" elements to the communication. For example, like-ability of subject, physical attractiveness or trust.

\paragraph{What are the uses of the peripheral route?}
The peripheral route is vastly used as a "cheap" route to convince people to perform an action. For example, buy a product, subscribe to a service or visit a location. It can be used to affect "central route" elaboration. It is especially effective when physical contact is not a factor. 

\paragraph{What are the six factors that affect likelihood of human persuasion?}
\begin{itemize}
	\item Reciprocation
	\item Consistency
	\item Social proof
	\item Liking
	\item Authority
	\item Scarcity
\end{itemize}

\paragraph{what is reciprocation (ELM)?}
Subjects will perform an action because that's socially customary. It is based on the notion of reciprocation of benefits. When subjects receive something they value, they feel "cognitive dissonance". We should keep in mind that promises count as something of value. Moreover, people tend to comply because they feel "gratitude" for the unsolicited proposal, it is socially frowned upon to not return a received favour.

\paragraph{What is consistency (ELM)?}
Subjects tend to maintain congruence in their attitudes and decisions even in presence of evidence that these are bad or turn out to be wrong/ineffective. Subjects tend to maintain cognitive consonance as opposed to face cognitive dissonance. In economics this is reflected in the concept of "loss aversion and sunk costs"

\paragraph{What is social proof (ELM)?}
People are influenced by the opinion and actions of those that are around them. Decisions of action taken to "mingle" within a clique or a circle of peers, even peers one does not know or have a direct relationship with, this concept is widely used in morning too. The emotional bond with interlocutor can be exploited to have the victim communicate personal details or perform certain actions. 

\paragraph{What is liking and trust (ELM)?}
People are willing to be liked by those whom they like, they will take action to obtain consent from those they like. People tend to extend "credibility" of subjects they perceive as successful beyond the reasonable boundaries of these subjects' actual expertise. When physical/presence attraction is not a factor, the like-ability can emerge from a "friendly connection"

\paragraph{What is authority (ELM)?}
People tend to respond to authority especially when in fear of the outcomes of not taking action. Obedience to authority is a very powerful tool to persuade people in executing actions or pertain to a certain behaviour. In some (occasionally very controversial) cases people will obey to authority even against well-established moral values and ethics.

\paragraph{What is scarcity (ELM)?}
 Scarcity leads people to take quick, potentially uniformed decisions in fear of losing an opportunity that will either disappear in time or that is scarce in quantity. It can be used by social engineers to elicit unwise decisions from the victim. 