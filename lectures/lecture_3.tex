\section{Transport layer}
\paragraph{How does traceroute work?}
Traceroute uses the TTL packet to see which hops it encounters. It gradually increases the TTL to see which hop will send back the ICMP error message.

\paragraph{What is a SYN DOS?}
When a server receives SYN J, it answers back with SYN K, ACK J+1. A server opens a new session in a separate thread, and allocated resources, for example for the transmission control block. A server then waits for the ACK K +1 from the client, the default maximum segment lifetime (MSL) is set to two minutes. However, the same mechanism is on the sender side, but because it is the attackers system, it can be abused.

Servers typically have more bandwidth available than a single client. Therefore, it is important that the client can drop all SYN ACKs to not exhaust its own memory, but throughput necessarily slows down by O(2n), for each SYN get a SYN ACK, which means a quickly decay in bandwidth. We must exhaust server's memory before throughput becomes insufficient.

A way around the throughput decay is to spoof an IP of another client. However, we need to make sure that the clients do not send an RST back to the server, which would clear up the allocated memory. If A then is fast enough, and the amount of SYN packets is big enough, then B will run out of memory.

\paragraph{What is TCP Session Hijacking?}
The goal of TCP session hijacking is that the attacker wants to send commands to a server they have no access to. The client it wants to hijack the session from is authorised with the server, for example by IP address authentication. The server must think that the attacker is the client, but the attacker does not sit between the client and the server.

A TCP segment between a client and a server is identified by
\begin{itemize}
	\item Client IP (known)
	\item Destination IP (known)
	\item Dst/src Port (known)
	\item Client SEQ number (known)
	\item Server SEQ number(unknown)
\end{itemize}

Hence, by means of sequence number prediction we can hijack the session.

\paragraph{What is a mitnick attack?}
In order to impersonate the client, the attacker only needs to correctly guess the server's SEQ number. There is a $\frac{1}{2^{32}}$ chance to get it right. In reality is is a lot simpler. We can predict the sequence number, and then run a DOS on our client such that it does not send a RST back to the server.

\paragraph{What is RIP v1?}
RIP v1 is a very simple hop-based routing protocol. Each router in the network will "advertise" the networks it links to and its distance to them. RIP v1 is unauthenticated, anybody can advertise a route to a network. Whichever route is the shortest, the other routers will forward traffic to it.

\paragraph{What is the difference between RIP v1 and RIP v2?}
Main difference of interest is authentication of the channel. Protocol accepts authentication mode, which is MD5. However, MD5 is not that hard to break.

\paragraph{What is Open Shortest Path First (OSPF)?}
OSPF is a modern extensively deployed Interior Gateway Protocol (IGP). It is a link state routing protocol, which means that it works at IP level. All routers have full map of the network, they independently calculate the shortest path, the shortest round trip path is then selected. Every router sends an announcement every 30 minutes that contains the following information:

\begin{itemize}
	\item Links to neighbouring networks/routers
	\item Cost to reach those links
\end{itemize}

These messages are called Link State Advertisement (LSA), a single LSA can contain more than one link. An announcement is flooded over the network by whom receives it. An LSA also contains the following attributes:
\begin{itemize}
	\item Sequence number: increase by 1 at new instance
	\item Age: how long ago was the entry generated
	\item Checksum: checksum of LSA, excluding the age field
	\item Link states: which links does the originating router advertise
\end{itemize}

However, an LSA is uniquely identified by (sequence number, age, checksum).

\paragraph{What are the security features of OSPF?}
OSPF fights-back, routers that receives a wrong announcement supposedly generated by itself, sends back a new announcement that corrects it. Bogus announcements are corrected.

Announcement flooding, all announcements are flooded to the entire network. 

Locality: each announcement has only local view of router, which makes it very difficult to poison the full topology.

Bidirectionally: a link is only valid if both ends advertise it.

The only way to attack this is to "fight back" and to win the race against the network. This way you can partially infect a part of the topology.