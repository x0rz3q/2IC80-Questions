\section{Pentesting as a process}
\paragraph{What are the different attackers and how resourceful are they?}
We first have the script kiddie, they mostly have a few scripted attacks, and cannot do much more than a simple DOS. They are mostly the media-like attacker for example anonymous. They are generally motivated by socio-political events or the like/dislike of certain corporate activities. They are technologically unsophisticated and rely on massive attacks as opposed to advanced attacks. They mostly target public facing systems. However, an updated system with good load balancing and a firewall should be able to minimise the impact. However, attacks can be, exceptionally, massive.

Next we have the general attacker, they have no specific payload they mostly drop general purpose malware on systems. A general attacker is mostly economically motivated. There are a varying degree of sophistication, but most attacks are one size fits all kind of deal. They often target older vulnerabilities. The vehicle of attack is generally some host system that .e.g visits an infected webpage.

The last one is the targeted attacker. For example, espionage where competitors are after information. Then there is nation-state, which is mostly related to tracking, surveillance and political aspects. The targeted attacker has an arbitrarily large motivation space, and are rarely opportunistic in nature. They are also arbitrarily sophisticated.

\paragraph{Explain human threats}
Employees are the most powerful internal threat. They can get distracted or can be unconcerned. Users can respond to phishing attempts. You can even have disgruntled ex-employees.