\section{Network reconnaissance and mapping}
\paragraph{What is a classic example of a bad attack?}
A bad attack is where you throw everything you have at the target(s). It is also called a "Hail Mary" attack. If you have enough attack power you will eventually find a matching attack. It is a very very bad idea, I mean very very very bad. Too many attacks implies that you will have a very high visibility. The goal of the attack is never to get noticed, not even for a denial of service attack.

\paragraph{What is a classic example of a good attack?}
A good attack first figures out how the network or system looks like. Which systems can you reach from your current location? What are the configurations of these systems? Which systems are more likely to function as a "bridge" to the rest of the network? Then, cherry pick the attacks you think are going to be the most successful.

\paragraph{What does a good defender do?}
A good defender will regularly perform system and network hardening to minimise the attack surface, it will test for working exploits and deploy network \& system deferences. They will apply the golden rule of defensive techniques, namely minimality principle. No user and no system component or process should be authorised to perform actions that are not strictly necessary for their normal operation. They will also implement network segmentation. This means that sensitive information can not directly be reached from public, unauthenticated positions. They will implement a network topology such that different functionalities are isolated from the rest.

\paragraph{What is a firewall?}
Firewalls are perimetric network components that filter incoming and outgoing traffic from and to the network. It is positioned such that all relevant traffic passes through them before entering/leaving the network. A set of rules define which traffic is allowed and which is not allowed.

\paragraph{What are the two policies of firewalls?}
Default permit: Allow everything EXCEPT what is defined the rules.

Default deny: Deny everything EXCEPT what is defined in the rules.

\paragraph{What firewall types are there?}
Stateless: Only looks at a single packet, e.g. the TCP flags, and has no history.

Stateful: Consider pre-existing states, e.g. SYN\_RECEIVED, ESTABLISHED) to admit meaningful packets.

Application firewalls: Deep packet inspection, evaluate L7 functionalities of protocol.

\paragraph{What is an IDS?}
IDS stands for intrusion detection system, it monitors incoming connections based on signatures and anomalies. We have two IDs systems, namely host IDS and network IDS. A host IDS is deployed on the host system, it monitors syscalls, system file hashing and system states. A Network IDS is deployed on the network. It does protocol analysis, it works similar to an application proxy. It is also the first that will fire an alarm.

On top of that, IDs, esp in large networks, generate many events. In the order of hundreds of thousands per day. Most of these events are false positives, it makes little sense when looked at on their own.

\paragraph{How do you evade NIDS?}
Evasion of signature-based detection can be fairly trivial, it will depend on the implementation of the actual signature. More advanced techniques are typically based on network features. All techniques common goal: NIDS sees different packet than client, the different ways are due to: TTL, fragmentation and timeouts.

The NIDS has a lower reassembly timeout than the receiving client. Which means, that fragment one can be delivered before fragment two is delivered. This way, the NIDS never sees the full picture.

The NIDS has a higher reassembly timeout than the receiving client. Which means, that fragment assembly can be deceived on the NIDS to trick it.

The router drops packet analysed by NIDS that will not be delivered to the victim. This happens due to the fact that the TTL is set accordingly, this way packets can be dropped and a new assembly can be created on the victim.

Another thing to keep in mind is that some systems replace fragments with newer ones, other keep old fragments instead of the new one.

\paragraph{What does good network segmentation allow?}
It allows for better management of firewall rules, higher control on the incoming traffic, higher overall security and lower load on single appliances.

\paragraph{What is a DNS zone transfer?}
A zone is a domain for which a server is authoritative. A slave server can ask a master (authoritative) server to copy their zone database. An attack then pretends to be a slave server and dumps the zone DB. This way the attacker acquires knowledge of the zone's infrastructure. It can be used to facilitate further attacks.

\paragraph{What port states are there?}
Open: An active server is listening to that port and will accept incoming connections.

Closed: No service is listening to the port, but the port responds to probes. It may be an index of misconfiguration, check back later to see if it goes to open.

Filtered: Cannot tell if open as probes don't get to the port. It may be an index of firewalling or other network filters between the host and the victim.

Unfiltered: Only know that it can be reached, but don't know if open or closed (ACK scan)

Open|Filtered: Does not know if open OR filtered, as the port does not provide any answer.

Closed|Filtered: same.

\paragraph{What is a SYN scan?}
The attacker forges TCP packets with the SYN flag up. It is useful to measure whether the remote system is accepting incoming connections on port x.

There is also a Half-open SYN scan. This means that the attacker will send an RST back after the server's SYNC ACK. This way, the 3-way handshake is never finished and control is not passed to the application layer.

\paragraph{What is an ACK scan?}
The attacker forges TCP packets with only the ACK flag up. It is useful to distinguish between stateful and stateless firewalls. A stateless firewall will let the packet go through even if there is no existing connections, the port needs to be open for this. A stateful firewall will drop the ACK packet if it is not associated with a SYN\_RECEIVED state. Therefore, we can conclude that the port is unfiltered when we receive an RST packet back. No response or specific ICMP message means that it is filtered.

\paragraph{What is an IDLE scan?}
An IDLE scan is useful to remain complete stealthy. It focusses on inferring the state of a remote port by looking at the state of a ZOMBIE machine. A zombie machine is a most-ly silent already existing system in the network. With an IDLE scan we can only see if a port is open, or if it is not, but we cannot distinguish between closed or filtered. We can also not get any OS fingerprinting information because we only have the IP state of the zombie machine to look at.

To see if a port is open or closed/filtered we do the following procedure:

First, the attacker needs to figure out the IP ID of the zombie. The attacker sends a SYN/ACK to a zombie. The zombie, which is not expecting a SYN/ACK sends back a RST disclosing its IP AD.

The next step is to forge a SYN packet from the zombie. The target sends a SYN/ACK in response to the SYN that appears to come from the zombie. The zombie, not expecting it, sends back a RST, incrementing its IP ID in the process. 

The third step is to probe the zombie's IP ID again. If it has increased by 2 then we know that the port is open. In case the IP ID has increased by 1, then we know that the port is either closed or filtered. We know this because the zombie will receive an RST from the target if the port is not open. Hence, not responding back. If the port would be open the zombie would send an RST to the target, as response to the SYN/ACK it receives from the target.

\paragraph{What is a FIN scan?} Set only the FIN flag.
\paragraph{What is a NULL scan?} Set all flags to 0
\paragraph{What is a XMAS scan?} Set the FIN, URG and PSH flag to one.
\paragraph{How does host fingerprinting work?}
Not all operating systems implement the RFC as stated. Moreover, this gives us the possibility to infer which operating system is on the other side on the basis of the received answer.